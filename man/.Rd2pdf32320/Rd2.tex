\documentclass[a4paper]{book}
\usepackage[times,inconsolata,hyper]{Rd}
\usepackage{makeidx}
\usepackage[utf8,latin1]{inputenc}
% \usepackage{graphicx} % @USE GRAPHICX@
\makeindex{}
\begin{document}
\chapter*{}
\begin{center}
{\textbf{\huge \R{} documentation}} \par\bigskip{{\Large of \file{waterbalance.Rd}}}
\par\bigskip{\large \today}
\end{center}
\inputencoding{utf8}
\HeaderA{waterbalance}{waterbalance Internal function for the water balance model}{waterbalance}
%
\begin{Description}\relax
waterbalance
Internal function for the water balance model
\end{Description}
%
\begin{Usage}
\begin{verbatim}
waterbalance(twilt, tfield, precipitation, GAI, date, ET0, L, alpha = 0.7)
\end{verbatim}
\end{Usage}
%
\begin{Arguments}
\begin{ldescription}
\item[\code{twilt}] wilting point (0 to 1)

\item[\code{tfield}] field capacity (0 to 1)

\item[\code{precipitation}] daily precipitations (mm)

\item[\code{GAI}] gren area index daily values

\item[\code{date}] date vector

\item[\code{ET0}] Evapotranspiration (calculated based on PET and GAI)

\item[\code{L}] soil depth (mm)
\end{ldescription}
\end{Arguments}
%
\begin{Details}\relax
The formulas come mainly from  Allen et al., 1998 \url{https://www.fao.org/3/x0490e/x0490e00.htm}
The calculation is done through multiple steps, iterated for each timestep:
\Tabular{lcc}{
Step & Description & Equation \\{}
0. & Soil water W is initialized assuming saturation, based on the depth L and volumetric capacity & \deqn{W[1] = \Theta_f*L }{} \\{}
1. & The single crop coefficient Kc is calculated based on GAI & \deqn{K_c=1.3-0.5*exp(-0.17*GAI)}{} \\{}
2. & calculation of crop evapotranspiration (ETc) under standard condition & \deqn{ET_c=ET_0*K_c}{} \\{}
3. & the intercepted water It is calculated based on crop ET, GAI and precipitation P & \deqn{It=min(P,ET_c,0.2*GAI)}{} \\{}
4. & potential evapotraspiration is calculated & \deqn{ E_{pot}=(ET_c-It)}{} \\{}
5. & Calculation of the percolation. Water (W\_b, water bypass) is lost when above field capacity, but allowing saturation for one day & \deqn{W_b = max(0, W-(\Theta_f*L))}{} \\{}
6. & Soil evaporation reduction coefficient & \deqn{Kr=max(0,(1-(0.9*tfield-\Theta)/(0.9*tfield-\alpha*twilt))^2)}{} \\{}
7. & Actual evapotraspiration is calculated & \deqn{E_{act}=E_{pot}*Kr }{} \\{}
8. & THe water balance is calculated (stepwise) & \deqn{ W[i+1]=W[i]+P[i]-E_{act}[i]-It-W_b[i]}{} \\{}
}


(Kr cannot be above one)
\end{Details}
%
\begin{Value}
The function returns a data frame with water balance and date (days)
\end{Value}
%
\begin{Author}\relax
Lorenzo Menichetti \email{ilmenichetti@gmail.com}
\end{Author}
\printindex{}
\end{document}
